\documentclass[11pt,a4paper]{article}
\usepackage{amsmath,amsthm,amsfonts,amssymb,amscd}
\usepackage{times}
\usepackage{graphicx}
\usepackage{float}
\usepackage{booktabs}
\usepackage{xcolor}
\usepackage{geometry}
\usepackage{wrapfig}
\usepackage{comment}
\usepackage{listings}
\usepackage{lastpage}
\usepackage{fancyhdr}
\usepackage{hyperref}
\usepackage[small,bf]{caption}
\usepackage{multirow}
\usepackage{algpseudocode}
\usepackage{algorithm}
\usepackage{multicol}
\usepackage{tikz}
\usepackage[section]{placeins}
\usepackage{verbatim}
\usepackage{cite}
\usepackage[us]{datetime}
\usepackage[utf8]{inputenc}
\usepackage{array}
\usepackage{makecell}
\usepackage{tabularx}
\usepackage{titlesec}
\usepackage{parskip}
\usepackage{caption}
\usepackage{subcaption}
\usepackage{nicematrix}
\usepackage{parskip}
\usepackage{url}
\usepackage{cleveref}
\usepackage{cancel}
\usepackage{pdfpages}
\usepackage{csvsimple}
\usepackage{tcolorbox}
% \usepackage[english]{babel}

\renewcommand{\t}[1]{\texttt{#1}}
\graphicspath{{../figure/}}
\input{defs.tex}
\begin{document}
\begin{titlepage}
    \noindent\makebox[\linewidth]{
        \begin{tikzpicture}
            \draw[UM_LightBlue!20, fill=UM_LightBlue!20] (-1,0)rectangle++(1.1\paperwidth,5);
            \node at (1\textwidth,2.5){\includegraphics[height=4.5cm]{McGill_University_CoA.svg.png}};
        \end{tikzpicture}
    }
    \vspace{40pt}
    \textcolor{UM_Brown}{
        \begin{flushleft}
            \textbf{\Huge Regression and ANOVA}\\
            \vspace{30pt}
            \Large Final Project \\
            \vspace{3pt}
            \Large MATH 533 \\
            \vspace{20pt}
            \large Authors: \textit{Ling Fei Zhang and Sevag Baghdassarian} \\
        \end{flushleft}
    }
    \vspace{180pt}
    \textcolor{UM_Brown}{
        \begin{flushright}
            \begin{tabular}{ll}
                Instructor: & \textit{Mehdi Dagdoug} \\
                Department: & Mathematics            \\
                Date:       & \today
            \end{tabular}
        \end{flushright}
        \hrule
    }
\end{titlepage}
\pagestyle{fancy}
\fancyhf{}
\rhead{Ling Fei Zhang, 260985358\\
    Sevag Baghdassarian, 260980928}
\lhead{\includegraphics[width=2.5cm]{mcgill.jpeg}}
\chead{MATH 533\\
    Final Project}
\cfoot{\thepage}
\algnewcommand\algorithmicforeach{\textbf{for each}}
\algdef{S}[FOR]{ForEach}[1]{\algorithmicforeach\ #1\ \algorithmicdo}

\tableofcontents
\newpage

\section{Introduction}

The housing market represents a complex interplay of various factors that
contribute to the determination of property prices. In this study, we
investigate into a comprehensive housing dataset, which captures a numerous of
covariates influencing house prices. The dataset encompasses essential features
such as the number of bedrooms, bathrooms, square footage, floor count, and
location details, among others.

\newpage

\section{Dataset Overview}
The dataset contains the following covariates: [id, date, price, bedrooms,
bathrooms, sqft living, sqft lot, floors, waterfront, view, condition, grade,
sqft above, sqft basement, yr built, yr renovated, zipcode, lat, long, sqft
living15, and sqft lot15]. While each of these variables holds potential
insights, certain attributes are deemed irrelevant for our analysis.
Specifically, the variables id, data, waterfront, view, condition, zipcode,
lat, long are excluded from our study. Additionally, to focus our analysis on a
more representative range of house prices, we have excluded houses with prices
exceeding \$2 million.

\newpage
\section{Objective of the Analysis}

 {\color{red}ADD TO THIS}

Our primary objective is to understand the intricate relationship between the
selected covariates and the housing prices. To achieve this, we employ three
distinct Linear Regression models. The models include a univariate model, which
explores the impact of \verb|sqft_living| on house prices; a bivariate model,
which explores the impact of \verb|sqft_living, yr_built| on prices; and a
multivariate model, which explores the effect of all interested covariates on
prices.

\newpage
\section{Model Visualizations}

\subsection{Univariate Model}

In the univariate case, we measure the relationship between the size of the
living room versus the price of the house. Below show the the result of our
model. In this plot, we've also added a confidence interval of 95\%.

{\color{red}EXPLAIN MORE}

\begin{figure}[H]
    \centering
    \includegraphics*[width=0.7\textwidth]{simple_regression.png}
    \caption{Univariate Linear Regression}
    \label{fig: univariate}
\end{figure}

\subsection{Bivariate Model}
In the bivariate case, we measure the relationship between the size of the
living room and the built year versus the price of the house. Below we show the
relationship of the covariates with the housing price via a 3D plot.
\begin{figure}[H]
    \centering
    \includegraphics*[width=0.7\textwidth]{multi_data.png}
    \caption{Bivariate Linear Regression}
    \label{fig: bivariate}
\end{figure}

\section{Model Summaries}

We display our models summary as tables below, following the format seen in
\verb|R|.

\subsection{Univariate Model}
\begin{table}[H]
    \centering
    \begin{subtable}{.7\linewidth}
        \centering
        \caption{Residuals}
        \csvautotabular{../figure/p1_residuals.csv}
    \end{subtable}%

    \vspace{1em} % Adjust the vertical space between subtables

    \begin{subtable}{.7\linewidth}
        \centering
        \caption{Coefficients}
        \csvautotabular{../figure/p1_coef_stats_table.csv}
    \end{subtable}%

    \vspace{1em}

    Residual standard error: 211022.60 on 21413 degrees of freedom

R-squared: 0.46, Adjusted R-squared: 0.46

F-statistic: 18422.07 on 1 and 21413 DF, p-value: 1.1102230246251565e-16



    \vspace{1em} % Adjust the vertical space between subtables

    \begin{subtable}{.7\linewidth}
        \centering
        \caption{Other Metrics}
        \csvautotabular{../figure/p1_other_metrics_table.csv}
    \end{subtable}

    \caption{Univariate Linear Regression Summary}
    \label{table: p1}
\end{table}

\subsection{Bivariate Model}
\begin{table}[H]
    \centering
    \begin{subtable}{.7\linewidth}
        \centering
        \caption{Residuals}
        \csvautotabular{../figure/p2_residuals.csv}
    \end{subtable}%

    \vspace{1em} % Adjust the vertical space between subtables

    \begin{subtable}{.7\linewidth}
        \centering
        \caption{Coefficients}
        \csvautotabular{../figure/p2_coef_stats_table.csv}
    \end{subtable}%

    \vspace{1em}

    Residual standard error: 204851.25 on 21412 degrees of freedom

R-squared: 0.49, Adjusted R-squared: 0.49

F-statistic: 10429.69 on 2 and 21412 DF, p-value: 1.1102230246251565e-16



    \vspace{1em} % Adjust the vertical space between subtables

    \begin{subtable}{.7\linewidth}
        \centering
        \caption{Other Metrics}
        \csvautotabular{../figure/p2_other_metrics_table.csv}
    \end{subtable}

    \caption{Bivariate Linear Regression Summary}
    \label{table: p2}
\end{table}

\subsection{Multi Variate Model}
\begin{table}[H]
    \centering
    \begin{subtable}{.7\linewidth}
        \centering
        \caption{Residuals}
        \csvautotabular{../figure/p13_residuals.csv}
    \end{subtable}%

    \vspace{1em} % Adjust the vertical space between subtables

    \begin{subtable}{.7\linewidth}
        \centering
        \caption{Coefficients}
        \csvautotabular{../figure/p13_coef_stats_table.csv}
    \end{subtable}%

    \vspace{1em}

    Residual standard error: 416593.71 on 21401 degrees of freedom

R-squared: -1.09, Adjusted R-squared: -1.10

F-statistic: 1850.62 on 13 and 21401 DF, p-value: 1.1102230246251565e-16



    \vspace{1em} % Adjust the vertical space between subtables

    \begin{subtable}{.7\linewidth}
        \centering
        \caption{Other Metrics}
        \csvautotabular{../figure/p13_other_metrics_table.csv}
    \end{subtable}

    \caption{Multi Variate Linear Regression Summary}
    \label{table: p13}
\end{table}

\newpage
\section{Hypothesis Testing}

We also performed hypothesis testing on our models, which are summarized as
tables below.

\subsection{Univariate Model}
\begin{table}[H]
    \centering
    \csvautotabular{../figure/p1_hypothesis_tests.csv}
    \caption{Hypothesis Testing in Univariate Linear Regression}
\end{table}

\subsection{Bivariate Model}
\begin{table}[H]
    \centering
    \csvautotabular{../figure/p2_hypothesis_tests.csv}
    \caption{Hypothesis Testing in Bivariate Linear Regression}
\end{table}

\subsection{Multi Variate Model}
\begin{table}[H]
    \centering
    \csvautotabular{../figure/p13_hypothesis_tests.csv}
    \caption{Hypothesis Testing in Multi Variate Linear Regression}
\end{table}

\newpage
\section{Conclusion}

\end{document}